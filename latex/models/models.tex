\documentclass[a4paper,12pt]{extreport}

\usepackage{extsizes}
\usepackage{cmap} % для кодировки шрифтов в pdf
\usepackage[T2A]{fontenc}
\usepackage[utf8]{inputenc}
\usepackage[english]{babel}

% \usepackage[usenames, dvipsnames]{color}
% \definecolor{fontColor}{RGB}{169, 183, 198}
% \definecolor{pageColor}{RGB}{43, 43, 43}

\usepackage{mathtools}

% \makeatletter
% \let\mytagform@=\tagform@
% \def\tagform@#1{\maketag@@@{\color{fontColor}(#1)}}
% \makeatother

%\renewcommand\theequation{{\color{fontColor}\arabic{equation_diff}}}

\linespread{1.3} % полуторный интервал
\renewcommand{\rmdefault}{ptm} % Times New Roman
\frenchspacing

\usepackage{graphicx}
\graphicspath{{images/}}
\usepackage{amssymb,amsfonts,amsmath,amsthm}
\usepackage{mathtext}
\usepackage{cite}
\usepackage{enumerate}
\usepackage{float}
% \usepackage[pdftex,unicode,colorlinks = true,linkcolor = white]{hyperref}
\usepackage[pdftex,unicode,colorlinks = true,linkcolor = black]{hyperref}
\usepackage{indentfirst}
\usepackage{placeins}
\bibliographystyle{unsrt}
\usepackage{makecell}
\usepackage{ulem}
\usepackage{longtable}
\usepackage{multirow}
\usepackage{multicol}

\usepackage{tikz}
\usetikzlibrary{arrows,decorations.pathmorphing,
	backgrounds,positioning,fit,petri}

\usepackage{fancyhdr}
\pagestyle{fancy}
\fancyhf{}
\fancyhead[R]{\thepage}
\fancyheadoffset{0mm}
\fancyfootoffset{0mm}
\setlength{\headheight}{17pt}
\renewcommand{\headrulewidth}{0pt}
\renewcommand{\footrulewidth}{0pt}
\fancypagestyle{plain}{
\fancyhf{}
\rhead{\thepage}}

\usepackage{geometry}
\geometry{left=1.5cm}
\geometry{right=1.5cm}
\geometry{top=2.4cm}
\geometry{bottom=2.4cm}

\author{Aleksandr Zhuravlyov}
\title{Numerical model of steady state and transient flow}
\date{\today}


\usepackage {titlesec}
\titleformat{\chapter}{\thispagestyle{myheadings}\centering\hyphenpenalty=10000\normalfont\huge\bfseries}{
\thechapter. }{0pt}{\Huge}
\makeatother


\usepackage{nomencl}
\makenomenclature    % Закомментируйте, если перечень не нужен
%"/usr/texbin/makeindex" %.nlo -s nomencl.ist -o %.nls
\renewcommand{\nomname}{Перечень условных обозначений}
\renewcommand{\nompreamble}{\markboth{}{}}
\newcommand*{\nom}[2]{#1~- #2\nomenclature{#1}{#2}}

\setlength{\columnseprule}{0.4pt}
\setlength{\columnsep}{50pt}
\def\columnseprulecolor{\color{fontColor}}


\begin{document}

%    \pagecolor{pageColor}
%    \color{fontColor}
    %\maketitle
    %\newpage
    %\tableofcontents{\thispagestyle{empty}}
    %\newpage

    \section*{One phase pore network model}

    \begin{eqnarray}
    \label{one_phase} 
    \begin{gathered}
    \sum^{N_{i}}_{j=1, \; i\neq j} G_{ij} = 0,
    \end{gathered}
    \end{eqnarray}
    %
    \begin{eqnarray}
    \begin{gathered}
    \label{press_bound}
    P_{m} = P^{in}   : m \in inlet, \;\;
    P_{m} = P^{out}   :m \in outlet,
    \end{gathered}
    \end{eqnarray}
    %
    \begin{eqnarray}
    \begin{gathered}
    \label{eq:mass_flux_simple_pnm}
    G_{ij} = \bar{\rho}_{ij} \frac{P_{i} - P_{j}}{R_{ij}},
    \end{gathered}
    \end{eqnarray}
    %
     \begin{eqnarray}
    \begin{gathered}
    \bar{\rho}_{ij} = \frac{\rho_{i}+\rho_{j}}{2},
    \end{gathered}
    \end{eqnarray}
    %
    \begin{eqnarray}
        \begin{gathered}
            \alpha_{ij} =\begin{cases}
                             1: &\text{pseudo compressible},\\
                             \frac{P_i + P_j}{2P_i}: &\text{ideal gas},
            \end{cases}
        \end{gathered}
    \end{eqnarray}
    %
    %
\begin{eqnarray}
\begin{gathered}
R_{ij} =\begin{cases}
\frac{8 \mu L_{ij}}{\pi r_{ij}^{4}}: &\text{cylindrical},\\
\frac{12 \mu L_{ij}}{\epsilon_{ij}^{3} \omega_{ij}}: &\text{plates},
\end{cases}
\end{gathered}
\end{eqnarray}
    %
\begin{eqnarray}
\begin{gathered}
G_{in/out} = \sum_{i}^{in/out} \sum^{N_{i}}_{j=1, \; i\neq j} G_{ij}.
\end{gathered}
\end{eqnarray}
    %
\begin{eqnarray}
\begin{gathered}
Q_{in/out} = \frac{G_{in/out}}{\rho_{in/out}}.
\end{gathered}
\end{eqnarray}
    %
\begin{eqnarray}
\begin{gathered}
k = a \frac{Q_{in/out}}{A} \frac{\mu L}{P_{in} - P_{out}},
\end{gathered}
\end{eqnarray}
    %
\begin{eqnarray}
\begin{gathered}
a =\begin{cases}
\frac{\rho_{in/out}}{\bar{\rho}}: &\text{pseudo compressible},\\
\frac{2P_{in/out}}{P_{in} + P_{out}}: &\text{ideal gas},
\end{cases}
\end{gathered}
\end{eqnarray}
    %
\begin{eqnarray}
\begin{gathered}
\bar{\rho}=\frac{1}{P_{in}-P_{out}} \int\limits^{P_{in}}_{P_{out}}\rho \; dP.
\end{gathered}
\end{eqnarray}

    \section*{FVM Diffusion transient flow}

   \begin{eqnarray}
   \label{eq:conductivity_integral}
   \int \limits_{V} \frac{\partial C}{\partial t} d V - \oint \limits_{S} D \vec{\nabla}C d\vec{\Omega} = 0,
   \end{eqnarray}
    \vspace{-0.5cm}
\begin{eqnarray}
\label{eq:conductivity_bound_integral}
C\left(t\right) \Big|_{cleat} = \hat{C} \left(t\right), \;
\vec{\nabla} C \Big|_{outer} = 0,
\end{eqnarray}
    %
    \begin{eqnarray}
    \label{eq:langmuir}
    \hat{C} = \frac{C_L\bar{P}}{P_L + \bar{P}}.
    \end{eqnarray}
    %
    \begin{eqnarray}
    \label{eq:conductivity_num}
    \alpha_i \Delta^{t}_i - \beta_{i-}\Delta^{n+1}_{i-} - \beta_{i+}\Delta^{n+1}_{i+}= 0,
    \end{eqnarray}
    \begin{eqnarray}
    \label{eq:conductivity_bound_num}
    C_1^n = \hat{C}^n, \; \Delta_{M+}^n = 0,
    \end{eqnarray}
    %
\begin{eqnarray}
\label{eq:alpha_beta}
\alpha_i = \frac{\Delta V_i}{\Delta t}, \;
\beta_{i\pm} = \frac{\overline{D}_{i\pm} \Delta \Omega_{i\pm}}{\Delta L_{i\pm}},
\end{eqnarray}
    %
\begin{eqnarray}
\Delta S_{i\pm} =\begin{cases}
h^2: & \text{cartesian},\\
2 \pi h \tilde{r}_{i\pm}: & \text{cylindrical},\\
4 \pi \tilde{r}^2_{i\pm}: & \text{spherical},
\end{cases}
\end{eqnarray}
%
\begin{eqnarray}
V=\begin{cases}
h^2 \left(\tilde{r}_{M+} - \tilde{r}_{1-}\right): & \text{cartesian},\\
\pi h \left(\tilde{r}_{M+}^2 - \tilde{r}_{1-}^2\right): & \text{cylindrical},\\
\frac{4}{3} \pi \left(\tilde{r}_{M+}^3-\tilde{r}_{1-}^3\right): & \text{spherical}.
\end{cases}
\end{eqnarray}
    %
\begin{eqnarray}
\label{eq:delta_num}
\Delta_{i\pm}^n = C_{i\pm1}^n - C_{i}^n, \;
\Delta_i^{t} = C_i^{n+1} - C_i^{n},
\end{eqnarray}
%
\begin{eqnarray}
\label{eq:Consumption_conductivity_integral}
\tilde{G}^{n+1} = -\sum_i^M \alpha_i \Delta_i^{t}.
\end{eqnarray}
\newpage
    \section*{PN and Diffusion coupled}

   \begin{eqnarray}
   \label{eq:main_coupled_sle}
   \begin{gathered}
   \sum^{N_{i}}_{j=1, \; i\neq j} G^{n+1}_{ij} = 0,
   \end{gathered}
   \end{eqnarray}
   %
   \begin{eqnarray}
   \label{eq:mass_flux_coupled}
   \begin{gathered}
   G^{n+1}_{ij} = \bar{\rho}_{ij}^n \frac{P^{n+1}_{i} - P^{n+1}_{j}}{R_{ij}} - \gamma^n_{ij} \tilde{G}^{n+1}_{ij},
   \end{gathered}
   \end{eqnarray}
    %
\begin{eqnarray}
\label{eq:gamma}
\gamma^n_{ij} =\begin{cases}
1: & G_{ij} < 0,\\
0: & G_{ij} \geqslant 0,
\end{cases}
\end{eqnarray}
    %
\begin{eqnarray}
\label{eq:bc_gamma}
\begin{gathered}
\sum^{N_{i}}_{j=1, \; i\neq j} G_{ij} = \hat{G} : i, j \notin outlet,\;\;
P_{m} = P^{out}   : m \in outlet,
\end{gathered}
\end{eqnarray}
%
\begin{eqnarray}
\label{eq:bc_coupled}
\begin{gathered}
\sum^{N_{i}}_{j=1, \; i\neq j} G^{n+1}_{ij} = G^{in}_{i} : i \in inlet,\;\;
P^{n+1}_{m} = P^{out}   : m \in outlet,
\end{gathered}
\end{eqnarray}
%
\begin{eqnarray}
\begin{gathered}
\tilde{G}_{ij}^{n+1} = \tilde{G}_{ij}'^n \cdot \left(\bar{P}_{ij}^{n+1}-\bar{P}_{ij}^n\right)+\tilde{G}_{ij}^n,
\end{gathered}
\end{eqnarray}
%
\begin{eqnarray}
\begin{gathered}
\tilde{G}_{ij}'^n = \frac{\tilde{G}_{ij}\left(\bar{P}_{ij}^n+ \frac{1}{2}\Delta P\right)-
	\tilde{G}_{ij}\left(\bar{P}_{ij}^n-\frac{1}{2}\Delta P\right)}{{\Delta P}},
\end{gathered}
\end{eqnarray}
%
\begin{eqnarray}
\begin{gathered}
\bar{P}^{n}_{ij} = \frac{P^n_i+P^n_j}{2}.
\end{gathered}
\end{eqnarray}
%
\begin{eqnarray}
\begin{gathered}
\tilde{G}_{ij}^{n+1} = \tilde{G}_{ij}^n.
\end{gathered}
\end{eqnarray}

 \section*{Two phase pore network model}
 
 \subsection*{Description} \label{s1}
 \begin{enumerate}
 	\item Solve for one-phase PN using Eqs. \ref{one_phase} and \ref{press_bound} and obtain the pressure distribution in each pore.
 	\item Knowing pressure in pores and throats, apply <<if statements>> (see Fig.~\ref{fig:workflow}) to get the displacement profile. Every iteration the phase (e.g., non-wetting) might proceed only to adjacent throats and pores.
 	\item Check for trapped throats every iteration. Remove the trapped throats from the calculation every iteration and recalculate the pressure distribution using one-phase PN. The trapped throats can be untrapped at further iterations.
 	\item Recalculate the state of each throat obeying “First In, Last Out” (FILO) logic. To preserve the mass-conservation law and to correctly treat the numerical time step, throats' states can be quasi-multiphase using FILO logic.
 \end{enumerate}
 
    \begin{eqnarray}
 	\begin{gathered}
 		\bar{\rho}_{ij} = \frac{\bar{\rho}_i+\bar{\rho}_{j}}{2},
 	\end{gathered}
 \end{eqnarray}
 %
   \begin{eqnarray}
 \begin{gathered}
 \bar{\rho}_{k} = S^w \rho_k^w+ S^g \rho_k^g,
 \end{gathered}
 \end{eqnarray}
 %
 \begin{eqnarray}
 \begin{gathered}
 \bar{\mu}_{ij} = \frac{\bar{\mu}_{i}+\bar{\mu}_{j}}{2},
 \end{gathered}
 \end{eqnarray}
 %
 \begin{eqnarray}
 \begin{gathered}
 \bar{\mu}_{k} = S^w \mu_k^w+ S^g \mu_k^g,
 \end{gathered}
 \end{eqnarray}


\begin{eqnarray}
\label{capill}
\begin{gathered}
F_{ij}^{c} = P_{ij}^c \epsilon_{ij}h_{ij},
\end{gathered}
\end{eqnarray}
where $F^{c}$ is a capillary force acting in a particular throat, $P^{c}$ is a capillary pressure exerted by film interface, $h$ is a fracture height.

\begin{eqnarray}
\label{capill}
\begin{gathered}
P_{ij}^{c} = \frac{2\sigma cos(\theta)}{R_{ij}},
\end{gathered}
\end{eqnarray}
where $\sigma$ is an interfacial tension, and $\theta$ is a contact angle.

\begin{eqnarray}
\label{drainage}
\begin{gathered}
F_{ij}^{f} = \frac{P_i - P_j}{L_{ij}} V_{ij}.
\end{gathered}
\end{eqnarray}
where index~$f=im, dr$, $F_f$ is the force provided by fluid flow, $V$ is a throat volume.

%\begin{figure}
%\include{two_phase_pnm.txt}	
%\end{figure}

\begin{figure}[ht!]
	
	\vspace{2 cm}
	\centering
	\input{two_phase_pnm.txt}
	
	\caption{Algorithmic Workflow}
	\label{fig:workflow}
\end{figure}

\end{document}
