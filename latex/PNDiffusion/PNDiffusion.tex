\documentclass[a4paper,12pt,russian]{extreport}
 
\usepackage{extsizes}
\usepackage{cmap} % для кодировки шрифтов в pdf
\usepackage[T2A]{fontenc}
\usepackage[utf8]{inputenc}
\usepackage[english, russian]{babel}

\usepackage[usenames, dvipsnames]{color}
\definecolor{fontColor}{RGB}{169, 183, 198}
\definecolor{pageColor}{RGB}{43, 43, 43}

\usepackage{mathtools}

\makeatletter
\let\mytagform@=\tagform@
\def\tagform@#1{\maketag@@@{\color{fontColor}(#1)}}
\makeatother

%\renewcommand\theequation{{\color{fontColor}\arabic{equation_diff}}}


  
\usepackage{graphicx} % для вставки картинок
\graphicspath{{images/}}
\usepackage{amssymb,amsfonts,amsmath,amsthm} % математические дополнения от АМС
\usepackage{mathtext}
\usepackage{cite}
\usepackage{enumerate}
\usepackage{float}
\usepackage[pdftex,unicode,colorlinks = true,linkcolor = white]{hyperref}
\usepackage{indentfirst} % отделять первую строку раздела абзацным отступом тоже
\usepackage{placeins}
\bibliographystyle{unsrt}
\usepackage{makecell}
\usepackage{multirow} % улучшенное форматирование таблиц
\usepackage{ulem} % подчеркивания
\usepackage{longtable} 
\usepackage{multirow}
\usepackage{multicol}
 
\linespread{1.3} % полуторный интервал
\renewcommand{\rmdefault}{ftm} % Times New Roman
\frenchspacing

\usepackage{fancyhdr}
\pagestyle{fancy}
\fancyhf{}
\fancyhead[R]{\thepage}
\fancyheadoffset{0mm}
\fancyfootoffset{0mm}
\setlength{\headheight}{17pt}
\renewcommand{\headrulewidth}{0pt}
\renewcommand{\footrulewidth}{0pt}
\fancypagestyle{plain}{ 
    \fancyhf{}
    \rhead{\thepage}}
    
\usepackage{geometry}
\geometry{left=1.5cm}
\geometry{right=1.5cm}
\geometry{top=2.4cm}
\geometry{bottom=2.4cm}

\author{Александр Сергеевич Журавлёв}
\title{Фильтрация\\Численные модели}
\date{\today}


\addto\captionsenglish{
\renewcommand\chaptername{Глава}
\renewcommand\contentsname{Содержание}
\renewcommand\figurename{Рис.}
\renewcommand\bibname{Литература}
}

\usepackage {titlesec}
\titleformat{\chapter}{\thispagestyle{myheadings}\centering\hyphenpenalty=10000\normalfont\huge\bfseries}{
\thechapter. }{0pt}{\Huge}
\makeatother


\usepackage{nomencl}
\makenomenclature    % Закомментируйте, если перечень не нужен
%"/usr/texbin/makeindex" %.nlo -s nomencl.ist -o %.nls
\renewcommand{\nomname}{Перечень условных обозначений}
\renewcommand{\nompreamble}{\markboth{}{}}
\newcommand*{\nom}[2]{#1~- #2\nomenclature{#1}{#2}}

\setlength{\columnseprule}{0.4pt}
\setlength{\columnsep}{50pt}
\def\columnseprulecolor{\color{fontColor}}


\begin{document}

\pagecolor{pageColor}
\color{fontColor}
\Russian
%\maketitle
%\newpage
%\tableofcontents{\thispagestyle{empty}}
%\newpage
\printnomenclature[5em]


\begin{center}
{\large \textbf{PN and Diffusion coupled}}
\end{center}

\begin{eqnarray}
\begin{cases}
\begin{gathered}
\sum^{N_{i}}_{j=1, \; i\neq j} q_{ij} = 0,   \\
q_{ij} = g_{ij} \left(P_{i} - P_{j}\right) + \hat{q}_{ij} ,
\end{gathered}
\end{cases}
\end{eqnarray}
%
\begin{eqnarray}
\begin{gathered}
\hat{q}_{ij} =\hat{q}_{ij} \left(\bar{P}_{ij}, t\right),  
\end{gathered}
\end{eqnarray}
%
\begin{eqnarray}
\begin{gathered}
\bar{P}_{ij} = \frac{P_i+P_j}{2}, 
\end{gathered}
\end{eqnarray}
%
\begin{eqnarray}
\begin{gathered}
\hat{q}_{ij}^{n+1} = \hat{q}_{ij}'^n\left(\bar{P}_{ij}^{n+1}-\bar{P}_{ij}^n\right)+\hat{q}_{ij}^n,
\end{gathered}
\end{eqnarray}
%
\begin{eqnarray}
\begin{gathered}
\hat{q}_{ij}^{m} = \hat{q}_{ij}\left(\bar{P}_{ij}^m\right),
\end{gathered}
\end{eqnarray}



\begin{eqnarray}
\begin{gathered}
\hat{q}_{ij}'^n = \frac{\hat{q}_{ij}\left(\bar{P}_{ij}^n+ \frac{1}{2}\Delta_P\right)-
\hat{q}_{ij}\left(\bar{P}_{ij}^n-\frac{1}{2}\Delta_P\right)}{{\Delta_P}}.
\end{gathered}
\end{eqnarray}
%

\end{document}

